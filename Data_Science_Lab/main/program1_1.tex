\mychapter{1}{Assignment 1 \\ \vspace{-0.3cm} Review of python programming}
\addcontentsline{toc}{chapter}{Assignment 1 Review of python programming}

\section*{Problem Statement}
\large
Write Python code to explore and practice with the basic data types, containers, functions, and classes of Python. 

\begin{enumerate}
    \item Start by creating variables of various numeric data types and assigning them values.
    \item Print the data types and values of these variables.
    \item Perform mathematical operations on these variables.
    \item Update the values of these variables.
    \item Create boolean variables with True or False values.
    \item Print the data types of these boolean variables.
    \item Perform Boolean operations on these boolean variables.
    \item Create string variables with text values.
    \item Print the contents and lengths of these string variables.
    \item Concatenate strings.
    \item Format strings with variables.
    \item Use string methods to manipulate strings by capitalizing, converting to uppercase, justifying, centering, replacing substrings, and stripping whitespace.
    \item Create and use Python lists. Perform tasks like appending elements, indexing, slicing, and iterating through the list.
    
    \item Create and use Python tuples. Perform tasks like indexing, slicing, and concatenation.

    \item Create and use Python sets. Perform tasks like accessing, adding, deleting set elements.
    
    \item Create and use Python dictionaries. Perform tasks like adding, updating, and removing key-value pairs, and accessing values.
    
    \item Define simple functions with parameters and return values.
    \item Call functions with different arguments and use the returned results.
    \item Write functions that accept other functions as arguments.
    \item Define and use Python classes. Include tasks like creating a class, defining methods, and creating instances.
    \item Implement class inheritance and method overriding.
    \item Create a class with class variables and instance variables, and demonstrate their usage.
\end{enumerate}









\section{Basic data types}
\vspace{-.15cm}
\subsection{Numbers}
\vspace{-.75cm}
\begin{code}
\begin{lstlisting}
x = 43
print(x)
print("Addition",x + 1)   
print("Subtraction",x - 1)   
print(" Multiplication",x * 2)   
print("Exponentiation",x ** 2)  
print("Division",x / 2)  
\end{lstlisting}
\end{code}
\vspace{-1cm}
\begin{verbatim}
43 <class 'int'> 
Addition 44
Subtraction 42
Multiplication 86
Exponentiation 1849
Division 21.5
\end{verbatim}


\vspace{-.6cm}
\subsection{Booleans}
\vspace{-.75cm}
\begin{code}
\begin{lstlisting}
t, f = True, False
print(type(t))
print(t and f) # Logical AND;
print(t or f)  # Logical OR;
print(not t)   # Logical NOT;
print(t != f)  # Logical XOR;
\end{lstlisting}
\end{code}
\vspace{-1cm}
\begin{verbatim}
<class 'bool'>
False True False True
\end{verbatim}
\subsection{Strings}
\vspace{-.75cm}
\begin{code}
\begin{lstlisting}
    hello='Hello'
    world='World'
    print(hello, len(hello))
    hw = hello + ' ' + world 
    print(hw)
    hw12 = '{} {} {}'.format(hello, world, 12) 
    print(hw12)
\end{lstlisting}
\end{code}
\vspace{-1cm}
\begin{verbatim}
hello 5
hello world
hello world 12

\end{verbatim}
\vspace{-.75cm}
\begin{code}
\begin{lstlisting}
   s = "hello"
   print(s.capitalize())
   print(s.upper()) 
   print(s.rjust(7))    
   print(s.center(7)) 
  print(s.replace('l', '(ell)'))
  print('  world '.strip())
\end{lstlisting}
\end{code}
\vspace{-1cm}
\begin{verbatim}
Hello
HELLO
 hello
   hello
he(ell)(ell)o
world
\end{verbatim}
\vspace{-.75cm}
\section{Containers}
\vspace{-.4cm}
\subsection{Lists}
\vspace{-.45cm}
\begin{code}
\begin{lstlisting}
li = [2, 3, 4, 5]
print(li, li[2])
print(li[-1]) 
li[2] = 'hai'
print(li)
li.append('hello')
print(li)
r = li.pop()
print(r, li)
\end{lstlisting}
\end{code}
\vspace{-.75cm}
\begin{verbatim}
[2, 3, 4, 5] 4
5
[2, 3, 'hai', 5]
[2, 3, 'hai', 5, 'hello']
hello [2, 3, 'hai', 5]
\end{verbatim}
%\vspace{-.75cm}
\newpage
\subsection{Slicing}
\vspace{-.6cm}
\begin{code}
\begin{lstlisting}
n = list(range(6))
print(n)
print(n[1:3])
print(n[3:])
print(n[:3])
print(n[:])
print(n[:-1])
n[2:4] = [8, 9]
print(n)
\end{lstlisting}
\end{code}
\vspace{-.75cm} 
\begin{verbatim}
[0, 1, 2, 3, 4,5]
[1, 2]
[0, 1, 2]
[3, 4, 5]
[0, 1, 2, 3, 4, 5]
[0, 1, 2, 3, 4]
[0, 1, 8, 9, 4, 5]
\end{verbatim}
\vspace{-.6cm}
\subsection{Loops}
\vspace{-.6cm}
\begin{code}
\begin{lstlisting}
animals = ['cat', 'dog', 'elephent']
for animal in animals:
    print(animal)
\end{lstlisting}
\end{code}
\vspace{-1cm}
\begin{verbatim}
cat
dog
elephent
\end{verbatim}
\vspace{-.6cm}
\subsection{List comprehensions}
\vspace{-.8cm}
\begin{code}
\begin{lstlisting}
num = [ 1, 2, 3, 4, 5]
sq = []
for i in num:
    sq.append(i ** 2)
print(sq)
\end{lstlisting}
\end{code}
\vspace{-1cm}
\begin{verbatim}
[ 1, 4, 9, 16, 25]
\end{verbatim}
\vspace{-.6cm}
\subsection{Dictionaries}
\vspace{-.6cm}
\begin{code}[H]
\begin{lstlisting}
d = {'cat': 'cute', 'dog': 'furry'}
print(d['cat'])
print('cat' in d)
d['fish'] = 'wet'
print(d['fish']) 
\end{lstlisting}
\end{code}
\vspace{-1.7cm}
\begin{verbatim}
cute
True
wet
\end{verbatim}
\newpage
\vspace{-.75cm}
\subsection{Sets}
\vspace{-.75cm}
\begin{code}
\begin{lstlisting}
animals = {'cat', 'dog'}
print('cat' in animals)
print('fish' in animals)
animals.add('cat')
print(len(animals))       
animals.remove('cat')
print(len(animals))
\end{lstlisting}
\end{code}
\vspace{-1cm}
\begin{verbatim}
True
False
3
2
\end{verbatim}
\vspace{-.75cm}
\subsection{Tuples}
\vspace{-.75cm}
\begin{code}
\begin{lstlisting}
d = {(x, x + 1): x for x in range(10)}
t = (5, 6)
print(type(t))
print(d[t])       
print(d[(1, 2)])
\end{lstlisting}
\end{code}
\vspace{-.95cm}
\begin{verbatim}
<class 'tuple'>
5
1
\end{verbatim}
\vspace{-.95cm}
\section{Functions}
\vspace{-.95cm}
\begin{code}[h]
\begin{lstlisting}
def sign(x):
    if x > 0:
        return 'positive'
    elif x < 0:
        return 'negative'
    else:
        return 'zero'
for x in [-1, 0, 1]:
    print(sign(x))
\end{lstlisting}
\end{code}
\vspace{-1cm}
\begin{verbatim}
negative
zero
positive
\end{verbatim}
\newpage
\vspace{-.95cm}
\begin{code}[h]
\begin{lstlisting}
def hello(name, loud=False):
    if loud:
        print("HELLO, {}".format(name.upper()))
    else:
         print("Hello, {}!".format(name))
hello("Anjana")
hello("Ram", loud=True)
\end{lstlisting}
\end{code}
\vspace{-1cm}
\begin{verbatim}
Hello, Anjana!
HELLO, RAM
\end{verbatim}
\vspace{-.6cm}
\begin{code}[h]
\begin{lstlisting}
def apply_function(func, value):
    return func(value)
def square(x):
    return x * x
def cube(x):
    return x * x * x
print(apply_function(square, 3)) 
print(apply_function(cube, 3))    
\end{lstlisting}
\end{code}
\vspace{-1cm}
\begin{verbatim}
9
27
\end{verbatim}
\section{Classes}
\vspace{-.75cm}
\begin{code}[h]
\begin{lstlisting}
class Greeter:
    def __init__(self, name):
        self.name = name
    def greet(self, loud=False):
        if loud:
          print('HELLO, {}'.format(self.name.upper()))
        else:
          print('Hello, {}!'.format(self.name))
g = Greeter('Fred')
g.greet()
g.greet(loud=True)
\end{lstlisting}
\end{code}
\begin{verbatim}
Hello, Fred!
HELLO, FRED
\end{verbatim}
\newpage
\subsection{Inheritance and Method overriding}
\vspace{-.6cm}
\begin{code}[h]
\begin{lstlisting}
class Animal:
    def __init__(self, name):
        self.name = name

class Dog(Animal):
    def speak(self):
        return f"{self.name} barks."

class Cat(Animal):
    def speak(self):
        return f"{self.name} meows."

print(Dog("Buddy").speak())  
print(Cat("Whiskers").speak())


\end{lstlisting}
\end{code}
\vspace{-.4cm}
\begin{verbatim}
Buddy barks.
Whiskers meows.
\end{verbatim}
\subsection{Class variables and Instance variables}
\vspace{-.6cm}
\begin{code}
\begin{lstlisting}
class MyClass:
    class_var = "I am a class variable"
    def __init__(self, instance_var):
        self.instance_var = instance_var
obj = MyClass("I am an instance variable")
print(MyClass.class_var)  
print(obj.class_var)     
print(obj.instance_var)   
\end{lstlisting}
\end{code}
\begin{verbatim}
    I am a class variable
    I am a class variable
    I am an instance variable

\end{verbatim}



 
