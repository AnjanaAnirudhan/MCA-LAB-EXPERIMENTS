\mychapter{1}{Assignment 2 \\ \vspace{-0.3cm} Vectorized Computations using Numpy}
\addcontentsline{toc}{chapter}{Assignment 2 Vectorized Computations using Numpy}

\section*{Problem Statement}
\large
Implement the following computations using NumPy: 

\begin{enumerate}
    \item Create a matrix U of shape (m, n) with input values where m and n are input
positive integers.
    \item Compute X as the transpose of U.
      \item Create a matrix $Y$ of shape $(1, m)$ with random values $\in [0, 1]$.
    \item Create a matrix $W1$ of shape $(p, n)$ with random values $\in [0, 1]$ where $p$ is an input positive integer.
    \item Create a vector $B1$ of shape $(p, 1)$ with random values $\in [0, 1]$.
    \item Create a vector $W2$ of shape $(1, p)$ with all zeros.
    \item Create a scalar $B2$ with a random value $\in [0, 1]$.
    \item Perform the following computations iteratively 15 times:
   \begin{enumerate}
        \item $Z1 = W1 \cdot X + B1$ \hspace{0.2cm} (Matrix Multiplication)
        \item $A1 = f(Z1)$ \hspace{0.2cm} where $f$ is a function that returns 0 for negative values and the input value itself otherwise.
        \item $Z2 = W2 \cdot A1 + B2$
        \item $A2 = g(Z2)$ \hspace{0.2cm} where $g$ is a function defined as $g(x) = \frac{1}{1+e^{-x}}$.
        \item $L = \frac{1}{2} (A2 - Y)^2$
        \item $dA2 = A2 - Y$
        \item $dZ2 = dA2 \circ gprime(Z2)$ \hspace{0.2cm} where $gprime(x)$ is a function that returns $g(x) \cdot (1 - g(x))$ and $\circ$ indicates element-wise multiplication.
        \item $dA1 = W2^T \cdot dZ2$
        \item $dZ1 = dA1 \circ fprime(Z1)$ \hspace{0.2cm} where $fprime$ is a function that returns 1 for positive values and 0 otherwise and $\circ$ indicates element-wise multiplication.
        \item $dW1 = \frac{1}{m} \cdot dZ1 \cdot X^T$
        \item $dB1 = \frac{1}{m} \sum dZ1$ \hspace{0.2cm} (sum along the columns)
        \item $dW2 = \frac{1}{m} \cdot dZ2 \cdot A1^T$
        \item $dB2 = \frac{1}{m} \sum dZ2$ \hspace{0.2cm} (sum along the columns)
        \item Update and print $W1$, $B1$, $W2$, and $B2$ for $\alpha = 0.01$:
        \begin{enumerate}
            \item $W1 = W1 - \alpha \cdot dW1$
            \item $B1 = B1 - \alpha \cdot dB1$
            \item $W2 = W2 - \alpha \cdot dW2$
            \item $B2 = B2 - \alpha \cdot dB2$
        \end{enumerate}
    \end{enumerate}
\end{enumerate}









\section{Matrix}
\vspace{-.15cm}
\subsection{Create Matrix}
\vspace{-.75cm}
\begin{code}
\begin{lstlisting}
import numpy as np
m=int(input("Enter row size:"))
n=int(input("Enter column size:"))
print("Enter values in single line(space seperated formst):")
entries=list(map(int,input().split()))
U=np.array(entries).reshape(m,n)
print(U)
\end{lstlisting}
\end{code}
\vspace{-1cm}
\begin{verbatim}
Enter row size: 3
Enter column size: 3
Enter values in single line(space seperated formst):
 12 3 -8 9 34 5 6 4 54
[[12  3 -8]
 [ 9 34  5]
 [ 6  4 54]]
\end{verbatim}
\vspace{-.6cm}
\subsection{Matrix Transpose}
\vspace{-.75cm}
\begin{code}
\begin{lstlisting}
X=U.T
print("Transpose of matrix U:")
print(X) 
\end{lstlisting}
\end{code}
\vspace{-1cm}
\begin{verbatim}
Transpose of matrix U:
[[12  9  6]
 [ 3 34  4]
 [-8  5 54]]
\end{verbatim}
\vspace{-.75cm}
\newpage
\subsection{Matrix with random values}
\vspace{-.75cm}
\begin{code}
\begin{lstlisting}
m=3
Y=np.random.rand(1,m)
print(Y) 

p=int(input("Enter  positive value,p :"))
n=3
W1=np.random.rand(p,n)
print("Matrix W1\n",W1)

p=int(input("Enter a value for p :"))
B1=np.random.rand(p,1)
print("Vector B1:")
print(B1)
\end{lstlisting}
\end{code}
\vspace{-1cm}
\begin{verbatim}
[[0.1017947  0.10587389 0.76135783]]
Enter  positive value,p : 3
Matrix W1
 [[0.12742882 0.59023487 0.70840949]
 [0.12882403 0.02896318 0.7421581 ]
 [0.87475144 0.40225054 0.97983924]]
 Enter a value for p : 3
Vector B1:
[[0.89954103]
 [0.9744558 ]
 [0.59537661]]
\end{verbatim}
\vspace{-.75cm}
\subsection{Matrix with zeros}
\vspace{-.6cm}
\begin{code}
\begin{lstlisting}
p=int(input("Enter value for p :"))
W2=np.zeros((1,p))
print(W2)

\end{lstlisting}
\end{code}
\vspace{-.75cm}
\begin{verbatim}
Enter value for p : 3
[[0. 0. 0.]]

\end{verbatim}
\vspace{-.6cm}
\subsection{Scalar matrix}
\vspace{-.6cm}
\begin{code}
\begin{lstlisting}
B2=np.random.rand()
print("Scalar B2:")
print(B2)
\end{lstlisting}
\end{code}
\vspace{-1cm}
\begin{verbatim}
Scalar B2:
0.16961376428131913
\end{verbatim}
\vspace{-.6cm}
\newpage
\subsection{Matrix Computations}

\begin{lstlisting}
import numpy as np

def f(Z):
    return np.maximum(0,Z)
    
def g(Z):
    return 1/(1+np.exp(-Z))
    
def gprime(Z):
    gz=g(Z)
    return gz*(1-gz)
    
def fprime(Z):
    return (Z > 0).astype(float)

alpha=0.01
for i in range(15):
    Z1=np.dot(W1,X) + B1

    A1=f(Z1)

    Z2=np.dot(W2,A1) + B2

    A2=g(Z2)

    L=0.5 * np.square(A2 - Y)

    dA2=A2 -Y

    dZ2=dA2 * gprime(Z2)

    dA1=np.dot(W2.T,dZ2)

    dZ1=dA1 * fprime(Z1)

    dW1=np.dot(dZ1, X.T)
\end{lstlisting}

\newpage

\begin{lstlisting}
    dB1=np.sum(dZ1, axis=1,keepdims=True)/m

    dW2=np.dot(dZ2,A1.T)/m

    dB2=np.sum(dZ2)/m

    W1 -=alpha * dW1
    B1 -=alpha * dB1
    W2 -=alpha * dW2
    B2 -=alpha * dB2
   
print("\n Updated W1:\n",W1)
print("\n Updated B1:\n",B1)
print("\n Updated W2:\n",W2)
print("\n Updated B2:\n",B2)
print("\n Loss L:\n",L)
\end{lstlisting}
\begin{verbatim}
Updated W1:
 [[0.12945891 0.60107285 0.70093643]
 [0.12389724 0.00263393 0.76036687]
 [0.88218076 0.41163383 0.97037548]]

 Updated B1:
 [[0.89959736]
 [0.97431926]
 [0.59558278]]

 Updated W2:
 [[-0.0333751   0.09194226 -0.03199405]]

 Updated B2:
 0.15579879317292614

 Loss L:
 [[0.08049264 0.01294096 0.0031798 ]]
\end{verbatim}